\subsection{Vending Machines 1}\label{sec-vm-1}

This is our first and simplest vending machine.

We begin with a protocol.
\begin{code}
Sales = protocol { *( coffee | tea | turn_off | ^service: Text ) }    
\end{code}

Then we define a vending machine that offers tea or coffee and can be turned off.
\begin{code}
VendingMachine = process numCoffee: Word; numTea: Word; rq: +Sales
{
    loop select
    {
        || rq.coffee; rq.service := 'a cup of coffee'
        || rq.tea; rq.service := 'a mug of tea'
        || rq.turn_off; exit
    }
}
\end{code}

Customers randomly request tea or coffee a few times then turn the machine off.
\begin{code}
Customer = process name: Text; numTries: Word; rq: -Sales
{
    for (t := 0; t <> numTries; t += 1)
    {
        r: Word := rand(2);
        assert 0 <= r and r <= 1;
        case r
        {
            |0| rq.coffee
            |1| rq.tea
        }
        scrln(name // ' gets ' // rq.service)
    }
    rq.turn_off
}
\end{code}

The vending machine and one customer are linked together\textellipsis
\begin{code}
Controller = cell
{
    rq: Sales;
    VendingMachine(10, 10, rq);
    Customer('Fred', 10, rq);
}
\end{code}
\textellipsis\ and executed.
\begin{code}
Controller()
\end{code}

Here's the output:
\begin{verbatim}
Fred gets a cup of coffee
Fred gets a mug of tea
Fred gets a cup of coffee
Fred gets a cup of coffee
Fred gets a mug of tea
Fred gets a cup of coffee
Fred gets a cup of coffee
Fred gets a mug of tea
Fred gets a cup of coffee
Fred gets a mug of tea
\end{verbatim}