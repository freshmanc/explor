\subsection{A Cluster of Servers --- with problems}

This version of the program is constructed as two clusters: a cluster of servers connected by a single channel to a cluster of clients.  Servers in the cluster must register with a manager.  This allows us to provide clean termination with any number of servers and clients.  \xf{fig-clus} shows the system that we will build in diagrammatic form.

\begin{figure}[htbp]
$$\includegraphics{fig-10.mps}$$
\caption{Two process clusters: servers and clients}
\label{fig-clus}
\end{figure}

There is a new protocol, \id{Register}, which servers use to communicate with their manager.
\begin{code}
NUMREQS = constant 10_000;
Register = protocol { register | terminate | start | stop }
Match = protocol { start; *(even | odd | ^reply: Word); stop }
\end{code}

A server must use the signal \id{register} to indicate its willingness to serve.  When the manager sends a \id{terminate} signal, the server quits.The \id{start} and \id{stop} signals from clients are forwarded to the manager.
\begin{code}
Server = process r: -Register; m: +Match 
{
  r.register;
  loop select 
    clients: Word := 0;
  {
    || m.start; r.start
    || m.stop; r.stop
    || r.terminate; exit
    || m.even; m.reply := 2 * rand(100)
    || m.odd; m.reply := 2 * rand(100) + 1
  }
}
\end{code}

The manager keeps counts of both servers and clients, using \id{register} and \id{start} signals, respectively.  When the last client sends \id{stop}, the manager sends one \id{terminate} signal for each server.
\begin{code}
Manager = process r: +Register
{
  servers: Word := 0;
  clients: Word := 0;
  loop select
  {
    || r.register; servers += 1
    || r.start; clients += 1
    || r.stop; clients -=1;
       case
       {
         |clients = 0|
            for (s := 0; s <> servers; s += 1)
              r.terminate;
            exit
       }
  }
}
\end{code}

The server cluster can have any number of servers, but is shown here with just one.
\begin{code}
Service = cell m: +Match
{
  r: Register;
  Manager(r);
  Server(r, m);
}
\end{code}

Clients are more or less the same as before.  They are now called \id{Even} and \id{Odd}, and are passed a unique name on start-up.
\begin{code}
Even = process name: Text; e: -Match
{
  e.start;
  evens: Word := 0;
  odds: Word := 0;
  for (r := 0; r < NUMREQS; r += 1)
  {
    e.even;
    n: Word := e.reply;
    case
    {
      |n % 2 = 0| evens += 1
      || odds += 1
    }
  }
  scrln(name // ': ' // evens // '  ' // odds);
  e.stop;
}
\end{code}

\begin{code}
Odd = process name: Text; o: -Match
{
  o.start;
  evens: Word := 0;
  odds: Word := 0;
  for (r := 0; r < NUMREQS; r += 1)
  {
    o.odd;
    n: Word := o.reply;
    case
    {
      |n % 2 = 0| evens += 1
      || odds += 1
    }
  }
  scrln(name // ': ' // odds // '  ' // evens);
  o.stop;
}
\end{code}

The customer cluster is a cell containing a number of clients connected to a single channel. 
\begin{code}
Customers = process m: -Match
{
   Even('Ann  ', m);
   Odd( 'Bill ', m);
   Even('Carol', m);
   Odd( 'David', m);
}
\end{code}

The program connects a collection of servers to a collection of clients.
\begin{code}
Main = cell
{
  m: Match;
  Service(m);
  Customers(m);
}

Main()
\end{code}

All is well with one server:
\begin{verbatim}
Carol: 10000  0
Ann  : 10000  0
David: 10000  0
Bill : 10000  0
\end{verbatim}

With two servers, however, problems arise.  Some clients get inappropriate replies:
\begin{verbatim}
Carol: 10000  0
Ann  : 9568  432
David: 9568  432
Bill : 10000  0
\end{verbatim}

Here is a cluster with five servers:
\begin{dummy}
Service = cell m: +Match
{
  r: Register;
  Manager(r);
  Server(r, m);
  Server(r, m);
  Server(r, m);
  Server(r, m);
  Server(r, m);
}
\end{dummy}

Running the program with this cluster gives even worse results:
\begin{verbatim}
Carol: 7140  2860
Ann  : 6966  3034
David: 7389  2611
Bill : 6717  3283
\end{verbatim}

The problem, of course, is the one that we described in the previous section: all replies have the same tag, \id{reply}, and may therefore get confused.

