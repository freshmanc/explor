\subsection{Registering Clients}

We still have the weakness of hard-wiring the number of clients into the code.  One way to avoid this is to require clients to register with the server.

In this version, a client must send the signal \id{start} to the server before sending requests, and must send \id{stop} when it has finished sending requests.
\begin{code}
Match = protocol { *(requestA |requestB |infoA |infoB |^resultA |^resultB) }

Server = process p: +Match
{
    loop select{
                ||p.requestA;p.infoA; p.resultA;
                ||p.requestB;p.infoB; p.resultB;
    }
}

Client = process e: -Match
{
  for (r := 0; r < 10000; r += 1)
  {
        case{
                ||e.requestA;e.infoA;e.resultA;
                ||e.requestB;e.infoB;e.resultB;
        }
  }
}

Main = cell
{
  m: Channel Match;
  Server(m);
  Client(m);
}

Main()
\end{code}

Here is the output from a typical run:
\begin{verbatim}
Steven: 1455  8545
Oddie:  7106  2894
\end{verbatim}

We observe that clients are not getting replies that match their requests.  This is because the server can distinguish requests (\id{even} or \id{odd}) but its replies (\id{reply}) can be accepted by either client.  Clearly, there are sequences in which both clients send requests and the replies get mixed up.

