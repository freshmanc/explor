\subsection{Vending Machines 3}\label{sec-vm-3}

The vending machine in this example avoids the problem of the previous machine by keeping a count of customers registered.  When a customer sends a \id{turn\_off} signal, the count is deccremented.  When it reaches zero, the vending machine shuts down.

We add \id{register} to the protocol for customer registration.
\begin{code}
Sales = protocol { *( register: Text | coffee | tea | turn_off | ^service: Text ) }    
\end{code}

The vending machine allows customers to register and maintains the number of registered customers.  Registration is not a requirement --- unregistered customers will be served --- but could be made so.
\begin{code}
VendingMachine = process numCoffee: Word; numTea: Word; rq: +Sales
{
    numCust: Word := 0;
    loop select
    {
        ||  name: Text := rq.register; 
            scrln(name // ' registered');
            numCust += 1
        ||  rq.coffee; 
            case
            {
                |numCoffee > 0| 
                    rq.service := 'a cup of coffee (' // numCoffee // ')';
                                numCoffee -= 1
                || rq.service := 'nothing (no coffee left)'
            }
        ||  rq.tea; 
            case
            {
                |numTea > 0| rq.service := 'a cup of tea (' // numTea // ')';
                                numTea -= 1
                || rq.service := 'nothing (no tea left)'
            }
        ||  rq.turn_off;
            numCust -= 1;
            case
            {
                |numCust = 0| 
                    scrln('Vending machine closing down');
                    exit;
            }
    }
}
\end{code}

Customers now register before using the machine.
\begin{code}
Customer = process name: Text; numTries: Word; rq: -Sales
{
    rq.register := name;
    for (t := 0; t <> numTries; t += 1)
    {
        r: Word := rand(2);
        assert 0 <= r and r <= 1;
        case r
        {
            |0| rq.coffee
            |1| rq.tea
        }
        scrln(name // ' gets ' // rq.service)
    }
    rq.turn_off;
    scrln(name // ' finished');
}
\end{code}

The rest of the code remains unchanged.
\begin{code}
Controller = cell
{
    rq: Sales;
    VendingMachine(15, 15, rq);
    Customer('Fred', 10, rq);
    Customer('Jane', 15, rq);
    Customer('Kate', 15, rq);
}
\end{code}

\begin{code}
Controller()
\end{code}

Here is an extract from the output from this program:
\begin{verbatim}
Fred registered
Jane registered
Kate registered
Fred gets a cup of coffee (15)
Jane gets a cup of coffee (14)
Jane gets a cup of coffee (2)
....
Jane gets a cup of coffee (1)
Kate gets nothing (no coffee left)
Jane gets nothing (no tea left)
Kate gets nothing (no coffee left)
Jane gets nothing (no coffee left)
Kate gets nothing (no coffee left)
Jane finished
Vending machine closing down
Kate finished
\end{verbatim}
