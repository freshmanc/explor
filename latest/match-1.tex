\subsection{Unconditional Response}

We define a constant, \id{NUMREQS}, that determines the length of the test, and a protocol, \id{Match}, that provides two kinds of request.  We expect an \id{odd} request to yield an odd \id{reply} and an \id{even} request to yield an even \id{reply}.
\begin{code}
NUMREQS = constant 10_000;
Match = protocol { even | odd | ^reply: Word }
\end{code}

The server responds to a request with a number of the appropriate parity.  It assumes (!) that there are two clients and terminates after processing $2 \times \id{NUMREQS}$ requests.
\begin{code}
Server = process p: +Match
{
  loop k: Word := rand(100); requests: Word := 0;
  {
    |requests = 2 * NUMREQS| exit
    ||
      requests += 1;
      select
      {
        || p.even; p.reply := 2 * k;
        || p.odd; p.reply := 2 * k + 1;
      }
  }
}
\end{code}

Process \id{Steven} requests even numbers and classifies the replies by parity.  The report gives the number of expected replies (even numbers) received followed by the count of unexpected replies (odd numbers) received.
\begin{code}
Steven = process e: -Match
{
  evens: Word := 0;
  odds: Word := 0;
  for (r := 0; r < NUMREQS; r += 1)
  {
    e.even;
    n: Word := e.reply;
    case
    {
      |n % 2 = 0| evens += 1
      || odds += 1
    }
  }
  scrln('Steven: ' // evens // '  ' // odds)
}
\end{code}

Process \id{Oddie} 
\footnote{Footlight \url{https://en.wikipedia.org/wiki/Bill_Oddie}}
is similar to \id{Steven} but expects odd numbers rather than even numbers.
\begin{code}
Oddie = process o: -Match
{
  evens: Word := 0;
  odds: Word := 0;
  for (r := 0; r < NUMREQS; r += 1)
  {
    o.odd;
    n: Word := o.reply;
    case
    {
      |n % 2 = 0| evens += 1
      || odds += 1
    }
  }
  scrln('Oddie:  ' // odds // '  ' // evens)
}
\end{code}

The program consists of a \id{Server} and one client of each parity.
\begin{code}
Main = cell
{
  m: Match;
  Server(m);
  Steven(m);
  Oddie(m);
}

Main()
\end{code}

The program generates the following output:
\begin{verbatim}
Steven: 10000 0
Oddie:  10000 0
\end{verbatim}

Why is it that \id{Steven} never receives an odd number and \id{Oddie} never receives an even number?  Suppose that both clients have issued a request and that the server has accepted \id{Steven}'s request.  When the server indicates that it has a reply ready, \id{Steven} is waiting for the reply but \id{Oddie} is waiting for its request to be processed.  Consequently, \id{Steven} gets the even number.