\documentclass{scrartcl}
\usepackage{cloe}
\usepackage{alltt}
\usepackage[pdftex,colorlinks=true,linkcolor=blue,citecolor=blue,urlcolor=blue]{hyperref}

\lstset{language=Desi}
\newcommand{\reptitle}{\desi: The Standard Prelude}

\begin{document}
\title\reptitle
\author{Peter Grogono and Brian Shearing}
\date{2 July 2015}
\maketitle

\begin{center}
	\begin{minipage}{4in}
    \tableofcontents
    \hfill
    $$\textbf{Revisions}$$
    \begin{tabular}{ll}
    150702 & Removed indexing commands. \\
    150702 & Removed map routines \id{delete} and \id{clear}. \\
    150623 & Changed external declarations to new form. \\
    150609 & Changed declarations for random-number generators. \\
    151221 & Changed declarations for generic routines. \\
    \end{tabular}
    \end{minipage}
\end{center}
\clearpage

$$\textbf{\Large\reptitle}$$

The Standard Prelude contains declarations for various types and routines provided by \desi.  The compiler reads the Prelude and uses the declarations for type checking and code generation.  Programmers can assume that anything declared in the Prelude can be used in a program without further action.

\section{Types}

\subsection{Enumerations}

Policies for \rw{select} statements are defined in the following enumeration.  This declaration must not be changed, because UDC assumes the values given here.  In particular, the default policy, \id{ordered}, has value 0.
\begin{code}
Policy = enum { ordered, fair, random }
\end{code}

These enumerations for stream input and output are used in \xs{sec-stream-io}:
\begin{code}
StreamMode = enum { In, Out, Append }
StreamState = enum { OK, OpenError, ReadError, WriteError }
\end{code}

\subsection{Type conversions}

Type conversions are defined as intrinsic operations.  Conversion routines must have one input argument and one output argument.

Most conversions generate the opcode ``\kw{c}'', indicated by defining the body of the routine as ``\kw{= \$c}''.  When this would be ambiguous (for example, when converting from \id{Real} to \id{Word}), a special opcode is used.

\subsection{Implicit Conversions}

The presence of the qualifier \rw{implicit} indicates that the compiler may generate calls to these routines whenever the context requires a conversion.  \udc\ never generates range checks since these will be performed by \jit.

Conversions from a type to itself are \emph{not} included.  \udc\ generates a 'cpy' command.  

To \id{Byte}:
\begin{code}
Byte = implicit routine w: Word -> b: Byte = $c;
\end{code}

To \id{Word}:
\begin{code}
Word = implicit routine b: Byte -> w: Word  = $c;
\end{code}

Values of an enumerated type may also be converted to \id{Word} implicitly.  Since the representation is the same, the \dil\ command is 'cpy'.  Generic type names must be unique, a restriction that will eventually be removed.

\begin{code}
Word = implicit routine<E: enum> e: E -> w: Word = $cpy;
\end{code}

To \id{Real}:
\begin{code}
Real = implicit routine b: Byte -> r: Real  = $c;
Real = implicit routine w: Word -> r: Real  = $c;
\end{code}

To \id{Text}:
\begin{code}
Text = implicit routine b: Byte -> t: Text = $c;
Text = implicit routine w: Word -> t: Text = $c;
Text = implicit routine r: Real -> t: Text = $c;
\end{code}

\subsection{Explicit Conversions}

The conversions in this section must be requested explicitly by the programmer for the reasons given.  For example, to convert the \id{Boolean} value in $b$ to \id{Word}, write \lstinline'Word(b)'.

There is a \dil\ opcode corresponding to each conversion: for this reason, these conversions are called \emph{intrinsic}.  For example, the body \lstinline'$flor' specifies that the compiler will generate the \dil\ opcode \kw{flor} to perform the conversion.

\begin{itemize}
	\item When converting from \id{Real} to \id{Word} the programmer must choose whether to round up (\id{ceil}), round down (\id{floor}), or round to the closest value (\id{round}).

	\item Implicit conversion from \id{Boolean} to \id{Word} would give the correct result --- but for the wrong reason:
	$$ 2 < 3 < 4 \longrightarrow
	   \id{Word}(\rw{true}) < 4 \longrightarrow
	   \rw{true} $$


	\item Implicit conversion from \id{Boolean} to \id{Text} would not even give a reasonable result:
	$$ 2 < 3 < 4 \longrightarrow
     \qt{true} < \qt{4} \longrightarrow
	   \rw{false} $$
\end{itemize}

\id{Real} to \id{Word}:
\begin{code}
floor = routine r: Real -> w: Word = $flor;
round = routine r: Real -> w: Word = $rnd;
ceil = routine r: Real -> w: Word = $ceil;
\end{code}

\id{Boolean} to \id{Word}:
\begin{code}
Word = routine b: Boolean -> w: Word = $c;
\end{code}

\id{Boolean} to \id{Text}:
\begin{code}
Text = routine b: Boolean -> t: Text = $c;
\end{code}

We provide a generic routine to convert protocols to \id{Word}.  The main use for this routine is debugging, and it is probably a security risk.  Generic type names must be unique, a restriction that will eventually be removed.
\begin{code}
Word = routine<P: protocol> e: P -> w: Word = $c;
\end{code}


\section{Operators}

All of the \desi\ operators are declared in this section.  If the body of the routine is simply \lstinline'$', \udc\ generates code to perform the operation.  If the body has the form \lstinline'$xxx', \udc\ emits the \dil\ opcode \kw{xxx}.

\subsection{Assignment}

\id{Boolean} operators other than copying are implemented by tests and branches.  The body of the routine is shown as \$, indicating that the declaration is used for type checking purposes only and that code is generated by \udc.

\subsubsection{\id{Boolean}}
\begin{code}
op := = routine e: Boolean -> v: Boolean = $cpy;
op and= = routine p: Boolean -> q: Boolean = $;
op implies= = routine p: Boolean -> q: Boolean = $;
op nand= = routine p: Boolean -> q: Boolean = $;
op nor= = routine p: Boolean -> q: Boolean = $;
op or= = routine p: Boolean -> q: Boolean = $;
op revimp= = routine p: Boolean -> q: Boolean = $;
op xor= = routine p: Boolean -> q: Boolean = $;
\end{code}

\subsubsection{\id{Byte}}

In a shift-and-assign operation such as \lstinline{w <<= n} (move the bits of $w$ left by $n$ bit positions), the left operand ($w$) must be a \rw{Word}.  If the right operand ($n$) is  a \rw{Byte}, it will be implicitly converted by \udc\ to \rw{Word} before the expression is evaluated.

\begin{code}
op := = routine e: Byte -> v: Byte = $cpy;
op += = routine e: Byte -> v: Byte = $add;
op -= = routine e: Byte -> v: Byte = $sub;
op *= = routine e: Byte -> v: Byte = $mul;
op /= = routine e: Byte -> v: Byte = $div;
op %= = routine e: Byte -> v: Byte = $rem;
op <<= = routine w: Word -> b: Byte = $lsh;
op >>= = routine w: Word -> b: Byte = $ash;
op >>>= = routine w: Word -> b: Byte = $rsh;
op andb= = routine p: Byte -> q: Byte = $and;
op impliesb= = routine p: Byte -> q: Byte = $imp;
op nandb= = routine p: Byte -> q: Byte = $nand;
op norb= = routine p: Byte -> q: Byte = $nor;
op orb= = routine p: Byte -> q: Byte = $or;
op revimpb= = routine p: Byte -> q: Byte = $rimp;
op xorb= = routine p: Byte -> q: Byte = $xor;
\end{code}

\subsubsection{\id{Word}}
\begin{code}
op := = routine e: Word -> v: Word = $cpy;
op += = routine e: Word -> v: Word = $add;
op -= = routine e: Word -> v: Word = $sub;
op *= = routine e: Word -> v: Word = $mul;
op /= = routine e: Word -> v: Word = $div;
op %= = routine e: Word -> v: Word = $rem;
op <<= = routine e: Word -> v: Word = $lsh;
op >>= = routine e: Word -> v: Word = $ash;
op >>>= = routine e: Word -> v: Word = $rsh;
op andb= = routine p: Word -> q: Word = $and;
op impliesb= = routine p: Word -> q: Word = $imp;
op nandb= = routine p: Word -> q: Word = $nand;
op norb= = routine p: Word -> q: Word = $nor;
op orb= = routine p: Word -> q: Word = $or;
op revimpb= = routine p: Word -> q: Word = $rimp;
op xorb= = routine p: Word -> q: Word = $xor;
\end{code}

\subsubsection{\id{Real}}
\begin{code}
op := = routine e: Real -> v: Real = $cpy;
op += = routine e: Real -> v: Real = $add;
op -= = routine e: Real -> v: Real = $sub;
op *= = routine e: Real -> v: Real = $mul;
op /= = routine e: Real -> v: Real = $div;
op %= = routine e: Real -> v: Real = $rem;
\end{code}

\subsubsection{\id{Text}}

\begin{code}
op := = routine x: Text -> y: Text = $cpy;
op //= = routine x: Text -> y: Text = $cat;
\end{code}

\subsection{Binary Operators}

\subsubsection{\id{Boolean}}

Calls to routines in this block generate a DIL operation with the
given name.
\begin{code}
op <   = routine x: Boolean; y: Boolean -> z: Boolean = $lt;
op <=  = routine x: Boolean; y: Boolean -> z: Boolean = $le;
op >=  = routine x: Boolean; y: Boolean -> z: Boolean = $ge;
op >   = routine x: Boolean; y: Boolean -> z: Boolean = $gt;
op =   = routine x: Boolean; y: Boolean -> z: Boolean = $eq;
op <>  = routine x: Boolean; y: Boolean -> z: Boolean = $ne;

op <=>  = routine x: Boolean; y: Boolean -> z: Boolean = $eq;
op iff  = routine x: Boolean; y: Boolean -> z: Boolean = $eq;
op xor  = routine x: Boolean; y: Boolean -> z: Boolean = $ne;
\end{code}

Calls to routines in this block generate code using conditional
branches.  There are no corresponding DIL operations.
\begin{code}
op and  = routine x: Boolean; y: Boolean -> z: Boolean = $;
op nand = routine x: Boolean; y: Boolean -> z: Boolean = $;
op or   = routine x: Boolean; y: Boolean -> z: Boolean = $;
op nor  = routine x: Boolean; y: Boolean -> z: Boolean = $;
op implies  = routine x: Boolean; y: Boolean -> z: Boolean = $;
op ==>  = routine x: Boolean; y: Boolean -> z: Boolean = $;
op revimp  = routine x: Boolean; y: Boolean -> z: Boolean = $;
op <==  = routine x: Boolean; y: Boolean -> z: Boolean = $;
\end{code}

\subsubsection{\id{Byte}}
\begin{code}
op + = routine x: Byte; y: Byte -> z: Byte = $add;
op - = routine x: Byte; y: Byte -> z: Byte = $sub;
op * = routine x: Byte; y: Byte -> z: Byte = $mul;
op / = routine x: Byte; y: Byte -> z: Byte = $div;
op % = routine x: Byte; y: Byte -> z: Byte = $rem;
\end{code}

\begin{code}
op andb = routine x: Byte; y: Byte -> z: Byte = $and;
op impliesb  = routine x: Byte; y: Byte -> z: Byte = $imp;
op nandb  = routine x: Byte; y: Byte -> z: Byte = $nand;
op norb  = routine x: Byte; y: Byte -> z: Byte = $nor;
op orb  = routine x: Byte; y: Byte -> z: Byte = $or;
op revimpb  = routine x: Byte; y: Byte -> z: Byte = $rimp;
op xorb = routine x: Byte; y: Byte -> z: Byte = $xor;
\end{code}

\begin{code}
op <  = routine x: Byte; y: Byte -> z: Boolean = $lt;
op <= = routine x: Byte; y: Byte -> z: Boolean = $le;
op >  = routine x: Byte; y: Byte -> z: Boolean = $gt;
op >= = routine x: Byte; y: Byte -> z: Boolean = $ge;
op =  = routine x: Byte; y: Byte -> z: Boolean = $eq;
op <> = routine x: Byte; y: Byte -> z: Boolean = $ne;
\end{code}

\begin{code}
op << = routine x: Byte; y: Word -> z: Byte = $lsh;
op >> = routine x: Byte; y: Word -> z: Byte = $ash;
op >>> = routine x: Byte; y: Word -> z: Byte = $rsh;
\end{code}

\subsubsection{\id{Word}}

\begin{code}
op + = routine x: Word; y: Word -> z: Word = $add;
op - = routine x: Word; y: Word -> z: Word = $sub;
op * = routine x: Word; y: Word -> z: Word = $mul;
op / = routine x: Word; y: Word -> z: Word = $div;
op % = routine x: Word; y: Word -> z: Word = $rem;
\end{code}

\begin{code}
op andb = routine x: Word; y: Word -> z: Word = $and;
op impliesb  = routine x: Word; y: Word -> z: Word = $imp;
op nandb  = routine x: Word; y: Word -> z: Word = $nand;
op norb  = routine x: Word; y: Word -> z: Word = $nor;
op orb  = routine x: Word; y: Word -> z: Word = $or;
op revimpb  = routine x: Word; y: Word -> z: Word = $rimp;
op xorb = routine x: Word; y: Word -> z: Word = $xor;
\end{code}

\begin{code}
op <  = routine x: Word; y: Word -> z: Boolean = $lt;
op <= = routine x: Word; y: Word -> z: Boolean = $le;
op >  = routine x: Word; y: Word -> z: Boolean = $gt;
op >= = routine x: Word; y: Word -> z: Boolean = $ge;
op =  = routine x: Word; y: Word -> z: Boolean = $eq;
op <> = routine x: Word; y: Word -> z: Boolean = $ne;
\end{code}

\begin{code}
op << = routine x: Word; y: Word -> z: Word = $lsh;
op >> = routine x: Word; y: Word -> z: Word = $ash;
op >>> = routine x: Word; y: Word -> z: Word = $rsh;
\end{code}

\subsubsection{\id{Real}}
\begin{code}
op + = routine x: Real; y: Real -> z: Real = $add;
op - = routine x: Real; y: Real -> z: Real = $sub;
op * = routine x: Real; y: Real -> z: Real = $mul;
op / = routine x: Real; y: Real -> z: Real = $div;
op % = routine x: Real; y: Real -> z: Real = $rem;
\end{code}

\begin{code}
op <  = routine x: Real; y: Real -> z: Boolean = $lt;
op <= = routine x: Real; y: Real -> z: Boolean = $le;
op >  = routine x: Real; y: Real -> z: Boolean = $gt;
op >= = routine x: Real; y: Real -> z: Boolean = $ge;
op =  = routine x: Real; y: Real -> z: Boolean = $eq;
op <> = routine x: Real; y: Real -> z: Boolean = $ne;
\end{code}

\subsubsection{\id{Text}}

The operator \kw{//} converts all legal operands to \id{Text}, whether or not there is an implicit conversion.
\begin{code}
op //  = routine x: Text; y: Text -> z: Text = $cat;
op <  = routine x: Text; y: Text -> z: Boolean = $lt;
op <= = routine x: Text; y: Text -> z: Boolean = $le;
op >  = routine x: Text; y: Text -> z: Boolean = $gt;
op >= = routine x: Text; y: Text -> z: Boolean = $ge;
op =  = routine x: Text; y: Text -> z: Boolean = $eq;
op <> = routine x: Text; y: Text -> z: Boolean = $ne;
\end{code}

\subsection{Unary Operators}

All unary operators are prefixed to their operands.  Unary minus is not defined for \id{Byte} operators, because there are no negative values of \id{Byte}.  The following code is allowed, the \id{Byte} being converted to \id{Word} before being negated:
\begin{dummy}
w: Word;  b: Byte;  w := -b;
\end{dummy}

The expression \lstinline'#t' yields the number of characters in the \id{Text t}.  The expression \lstinline'~b' yields the \id{Byte} or \id{Word} $b$ with its bits inverted.

\begin{code}
op not = routine x: Boolean -> y: Boolean = $not;

op + = routine x: Byte -> y: Byte = $cpy;
op + = routine x: Word -> y: Word = $cpy;
op + = routine x: Real -> y: Real = $cpy;

op - = routine x: Word -> y: Word = $neg;
op - = routine x: Real -> y: Real = $neg;
op # = routine t: Text -> w: Word = $len;

op ~ = routine b: Byte -> c: Byte = $inv;
op ~ = routine b: Word -> c: Word = $inv;
\end{code}

\subsection{Map Operators}

The expression \lstinline'#m' yields the number of elements in the map $m$.  Generic type names must be unique, a restriction that will eventually be removed.
\begin{code}
op # = routine<D->R> m: D indexes R -> w: Word = $len;
\end{code}

\section{External Routines}

A routine body with the form ``\rw{external} \textsf{TextLiteral}'' informs the compiler that there is a routine with the given name and parameters available to the run-time system and that it was written in the language specified by the \id{Text} literal.

When a \desi\ program invokes an external routine, the compiler checks the types of the arguments, introduces coercions if necessary, and emits a \kw{call} instruction.

\udc\ uses the definitions in this section solely for type-checking.  Consequently, they can be changed, and new definitions can be added, without modifying \udc.

\subsection{Basic Input and Output}

The following routines are proposed for handling the keyboard and display.  The names may be changed according to taste.  
\begin{code}
kbdch = routine -> t: Text external "C", "kbdch";
scrch = routine t: Text external "C", "scrch";

kbdln = routine -> t: Text external "C", "kbdln";
scrln = routine t: Text external "C", "scrln";
\end{code}

\subsection{Stream Input and Output}
\label{sec-stream-io}

The following routines are proposed for stream processing.  The names and implementations of these routines can be changed arbitrarily, because \udc\ knows nothing about them.
\begin{code}
open  = routine path: Text; mode: StreamMode -> str: Stream external "C", "open";
close = routine str: Stream external "C", "close";
state = routine str: Stream -> sta: StreamState external "C", "state";
eof   = routine str: Stream -> end: Boolean external "C", "eof";
read  = routine str: Stream -> line: Text external "C", "read";
write = routine str: Stream; line: Text external "C", "write";
\end{code}

\subsection{Mathematical functions}

The routines in this section are intended to suggest a starting point for a comprehensive mathematical library.
\begin{code}
abs = routine x: Real -> y: Real external "C", "abs_real";
abs = routine x: Word -> y: Word external "C", "abs_word";
sqrt = routine x: Real -> y: Real external "C", "sqrt_real";
exp = routine x: Real -> y: Real external "C", "exp_real";
log = routine x: Real -> y: Real external "C", "log_real";
sin = routine x: Real -> y: Real external "C", "sin_real";
cos = routine x: Real -> y: Real external "C", "cos_real";
tan = routine x: Real -> y: Real external "C", "tan_real";
atan = routine x: Real -> y: Real external "C", "atan_real";
atan2 = routine y, x: Real -> z: Real external "C", "atan2_real_real";
\end{code}

\subsection{Random number generators}

\desi\ would allow these routines to have the same name, choosing between them on the basis of parameter types.  Since the convention that we have currently adopted requires the \desi\ name and the external name to be the same, however, we use distinct names.

Generate a uniformly distributed random \id{Word} in $[0,m)$:
\begin{code}
rand = routine m: Word -> r: Word external "C", "rand_word";
\end{code}

Generate a uniformly distributed random \id{Real} in $[0,x)$:
\begin{code}
rand = routine x: Real -> r: Real external "C", "rand_real";
\end{code}

Generate a uniformly distributed random \id{Real} in $[0,1)$:
\begin{code}
rand = routine -> r: Real external "C", "rand_unit";
\end{code}

Set the random number generator seed.
\begin{code}
seed = routine s: Word external "C", "rand_seed"; 
\end{code}

Protocols corresponding to services provided by the operating system.
\begin{code}
Screen = protocol { *(out: Text | outln: Text) }
Keyboard = protocol { *(char: Text | line: Text) }
OS = protocol {*(^scr: Client Screen | ^kbd: Server Keyboard) }
\end{code}

\end{document}
